To find a model of the system we started with Newton's 2nd law for rotation, which states that
\begin{equation} \label{eq:N2rot}
\sum \tau = J \alpha
\end{equation}
where $\tau$ is the external torque, $J$ is the moment of inertia, and $\alpha$ is the angular acceleration. Using this for each of the three axis gives
\begin{subequations}
  \begin{align}
    J_p\ddot{p} &= L_{1}V_{d} \label{eq:ulinsysp}\\
    J_e\ddot{e} &= L_{2} \cos(e) + L_3 V_s \cos(p) \label{eq:ulinsyse}\\
    J_\lambda \ddot{\lambda} &= L_4 V_s \cos(e) \sin(p) \label{eq:ulinsysl}
  \end{align}
  \label{eq:ulinsys}
\end{subequations}
where 
\begin{subequations}
	\begin{align*}
		L_1 &= K_f l_p\\
		L_2 &= (m_c l_c - 2 m_p l_h)g\\
		L_3 &= K_f l_h
	\end{align*}
\end{subequations}

As the model in \cref{eq:ulinsys} is non-linear we need to linearize the model to be able to design a linear controller. To do this we need a point to linearize around. For this we use $(p^*, e^*, \lambda^*)^T = (0, 0, 0)^T$. We also need to find the voltages $V_s^*$ and $V_d^*$ that makes $(p^*, e^*, \lambda^*)^T$ an equilibrium. Setting \cref{eq:ulinsysp} and \cref{eq:ulinsyse} to zero gives

\begin{subequations}
	\begin{align*}
		V_d^* &= 0\\
		V_s^* &= -\frac{L_2}{L_3}
	\end{align*}
\end{subequations}

The next thing we did was a coordinate transform, to simplify the model of the system. The new states are $ (\tilde p, \tilde e, \tilde \lambda)^T = (p, e, \lambda)^T - (p^*, e^*, \lambda^*)^T $ and the new inputs are $(\tilde V_s, \tilde V_d)^T = (V_s, V_d)^T - (V_s^*, V_d^*)^T $. This gives the system

\begin{subequations}
	\begin{align}
		\ddot{\tilde p} &= \frac{L_1}{J_p} \tilde V_d\\
		\ddot{\tilde e} &= \frac{L_2}{J_e} \cos \tilde e + (L_3 \tilde V_s - L_2) \cos \tilde p\\
		\ddot{\tilde \lambda} &= -\frac{L_2}{J_{\lambda}} \tilde p
	\end{align}
	\label{eq:transformertulinsys}
\end{subequations}


Now that the system is on a nice form, we can linearize it. This gives us a system on the form $ \dot{\mbf{x}} = \mbf{A} \mbf{x} + \mbf{B} \mbf {u} $, where $\mbf{\dot x} = (\ddot{\tilde p}, \ddot{\tilde e}, \ddot{\tilde \lambda})^T$, and the matrices $\mbf{A}$ and $\mbf{B}$ is given by
	
\begin{subequations}
	\begin{align}
		\mbf{A} =\left. \begin{bmatrix}
			\partialderiv{\ddot{\tilde{p}}}{\tilde p} & \partialderiv{\ddot{\tilde p}}{\tilde e} & \partialderiv{\ddot{\tilde p}}{\tilde \lambda}\\[0.3em]
			\partialderiv{\ddot{\tilde{e}}}{\tilde p} & \partialderiv{\ddot{\tilde{e}}}{\tilde e} & \partialderiv{\ddot{\tilde{e}}}{\tilde \lambda}\\[0.3em]
			\partialderiv{\ddot{\tilde \lambda}}{\tilde p} & \partialderiv{\ddot{\tilde \lambda}}{\tilde e} & \partialderiv{\ddot{\tilde \lambda}}{\tilde \lambda}
		\end{bmatrix} \right|_{\ddot{\tilde p} = 0 ,\, \ddot{\tilde e} = 0,\, \ddot{\tilde \lambda} = 0 } &= 
		\begin{bmatrix}
			0 & 0 & 0\\[0.3em]
			0 & 0 & 0\\[0.3em]
			-\frac{L_2}{J_\lambda} & 0 & 0
		\end{bmatrix}\\
		\mbf{B} = \left. \begin{bmatrix}
			\partialderiv{\ddot{\tilde p}}{\tilde V_s} & \partialderiv{\ddot{\tilde p}}{\tilde V_d}\\[0.3em]
			\partialderiv{\ddot{\tilde e}}{\tilde V_s} & \partialderiv{\ddot{\tilde e}}{\tilde V_d}\\[0.3em]
			\partialderiv{\ddot{\tilde \lambda}}{\tilde V_s} & \partialderiv{\ddot{\tilde \lambda}}{\tilde V_d}
		\end{bmatrix} \right|_{\tilde V_s = 0,\, \tilde V_d = 0} &= 
		\begin{bmatrix}
			0 & \frac{L_1}{J_p}\\[0.3em]
			\frac{L_3}{J_e} & 0\\[0.3em]
			0 & 0
		\end{bmatrix}
	\end{align}
	\label{eq:linearisert}
\end{subequations}

Expanding \cref{eq:linearisert} we get

\begin{subequations}
	\begin{align}
		\ddot{\tilde p} &= \frac{L_1}{J_p} \tilde V_d = K_1 \tilde V_d\\
		\ddot{\tilde e} &= \frac{L_3}{J_e} \tilde V_s = K_2 \tilde V_s\\
		\ddot{\tilde \lambda} &= -\frac{L_2}{J_{\lambda}} \tilde p = K_3 \tilde p \label{eq:linddotlambda}
	\end{align}
\end{subequations}

\iffalse
Linearizing the system gives

\begin{subequations}
	\begin{align}
		\ddot{\tilde p} &= \partialderiv{\ddot{\tilde{p}}}{\tilde p} + \partialderiv{\ddot{\tilde p}}{\tilde e} + \partialderiv{\ddot{\tilde p}}{\tilde \lambda} + \partialderiv{\ddot{\tilde p}}{\tilde V_s} + \partialderiv{\ddot{\tilde p}}{\tilde V_d} = 0 + 0 + 0 + 0 + \frac{L_1}{J_p} \tilde V_d = \frac{L_1}{J_p} \tilde V_d = K_1 \tilde V_d\\
		\ddot{\tilde e} &= \partialderiv{\ddot{\tilde{e}}}{\tilde p} + \partialderiv{\ddot{\tilde{e}}}{\tilde e} + \partialderiv{\ddot{\tilde{e}}}{\tilde \lambda} + \partialderiv{\ddot{\tilde e}}{\tilde V_s} + \partialderiv{\ddot{\tilde e}}{\tilde V_d} = 0 + 0 + 0 + \frac{L_3}{J_e} \tilde V_s + 0 = \frac{L_3}{J_e} \tilde V_s = K_2 \tilde V_s\\
		\ddot{\tilde \lambda} &= \partialderiv{\ddot{\tilde \lambda}}{\tilde p} + \partialderiv{\ddot{\tilde \lambda}}{\tilde p} + \partialderiv{\ddot{\tilde \lambda}}{\tilde \lambda} + \partialderiv{\ddot{\tilde \lambda}}{\tilde V_s} + \partialderiv{\ddot{\tilde \lambda}}{\tilde V_d} = -\frac{L_2}{J_{\lambda}} \tilde p + 0 + 0 + 0 + 0 = -\frac{L_2}{J_{\lambda}} \tilde p = K_3 \tilde p 
	\end{align}
\end{subequations}
\fi

Using this model we tried to fly the helicopter using only feed-forward, without any regulator. As expected it was very hard to keep the helicopter stable for more than a few seconds. The helicopter was behaving especially bad with a large elevation, something which according to the linearized model should not be the case. The reason for this is that at a high elevation the force generated by the propellers aren't vertical, and as such the motors need to provide a bigger force to keep the helicopter elevated. This is done by increasing the voltage given to the motors. However the motors can only take so much voltage, and at a high elevation, the motors are  close to their's saturation limit. This makes it hard to control the pitch (and travel) without sacrificing elevation.

After this we did some more preparations to be able to control the helicopter. First we added constant values the the output of the encoder so that we would be able to start the helicopter form the table, rather than from all joints at zero. Next we found the motor force constant $K_f$ by finding the voltage that would keep the helicopter horizontal. $K_f$ was found to be $0.14689$ \todo{gjorde vi noen utregninger her?}