To find a model of the system we started with Newton's 2nd law for rotation, which states that
\begin{equation} \label{eq:N2rot}
\sum \tau = J \alpha
\end{equation}
where $\tau$ is the external torque, $J$ is the moment of inertia, and $\alpha$ is the angular acceleration. Using this for each of the three axis gives
\begin{subequations}
  \begin{align}
    J_p\ddot{p} &= L_{1}V_{d} \label{eq:ulinsysp}\\
    J_e\ddot{e} &= L_{2} \cos(e) + L_3 V_s \cos(p) \label{eq:ulinsyse}\\
    J_\lambda \ddot{\lambda} &= L_4 V_s \cos(e) \sin(p) \label{eq:ulinsysl}
  \end{align}
  \label{eq:ulinsys}
\end{subequations}
where 
\begin{subequations}
	\begin{align*}
		L_1 &= K_f l_p\\
		L_2 &= (m_c l_c - 2 m_p l_h)g\\
		L_3 &= K_f l_h
	\end{align*}
\end{subequations}

As the model in \cref{eq:ulinsys} is non-linear we need to linearize the model to be able to design a linear controller. To do this we need a point to linearize around. For this we use $(p^*, e^*, \lambda^*)^T = (0, 0, 0)^T$. We also need to find the voltages $V_s^*$ and $V_d^*$ that makes $(p^*, e^*, \lambda^*)^T$ an equilibrium. Setting \cref{eq:ulinsysp} and \cref{eq:ulinsyse} to zero gives

\begin{subequations}
	\begin{align*}
		V_d^* &= 0\\
		V_s^* &= -\frac{L_2}{L_3}
	\end{align*}
\end{subequations}

The next thing we did was a coordinate transform, to simplify the model of the system. The new states are $ (\tilde p, \tilde e, \tilde \lambda)^T = (p, e, \lambda)^T - (p^*, e^*, \lambda^*)^T $ and the new inputs are $(\tilde V_s, \tilde V_d)^T = (V_s, V_d)^T - (V_s^*, V_d^*)^T $. This gives the system

\begin{subequations}
	\begin{align}
		\ddot{\tilde p} &= \frac{L_1}{J_p} \tilde V_d\\
		\ddot{\tilde e} &= \frac{L_2}{J_e} \cos \tilde e + (L_3 \tilde V_s - L_2) \cos \tilde p\\
		\ddot{\tilde \lambda} &= %\todo{sjekk notater til Bernt Johan om utrykket står der}
	\end{align}
	\label{eq:transformertulinsys}
\end{subequations}

Now that the system is on a nice form, we can linearize it. This gives us a system on the form $ \dot{\mbf{x}} = \mbf{A} \mbf{x} + \mbf{B} \mbf {u} $, where the matrices is given by
	
\begin{subequations}
	\begin{align}
		\mbf{A} &= \begin{bmatrix}
			\partialderiv{\ddot{\tilde{p}}}{\tilde p} = 0 & \partialderiv{\ddot{\tilde p}}{\tilde e} = 0 & \partialderiv{\ddot{\tilde p}}{\tilde \lambda} = 0\\[0.3em]
			\partialderiv{\ddot{\tilde{e}}}{\tilde p} = 0 & \partialderiv{\ddot{\tilde{e}}}{\tilde e} = 0 & \partialderiv{\ddot{\tilde{e}}}{\tilde \lambda} = 0\\[0.3em]
			\partialderiv{\ddot{\tilde \lambda}}{\tilde p} = -\frac{L_2}{J_{\lambda}} & \partialderiv{\ddot{\tilde \lambda}}{\tilde e} = 0 & \partialderiv{\ddot{\tilde \lambda}}{\tilde \lambda} = 0
		\end{bmatrix} \\
		\mbf{B} &= \begin{bmatrix}
			\partialderiv{\ddot{\tilde p}}{\tilde V_s} = 0 & \partialderiv{\ddot{\tilde p}}{\tilde V_d} = \frac{L_1}{J_p}\\[0.3em]
			\partialderiv{\ddot{\tilde e}}{\tilde V_s} = \frac{L_3}{J_e} & \partialderiv{\ddot{\tilde e}}{\tilde V_d} = 0\\[0.3em]
			\partialderiv{\ddot{\tilde \lambda}}{\tilde V_s} = 0 & \partialderiv{\ddot{\tilde \lambda}}{\tilde V_d} = 0
		\end{bmatrix}
	\end{align}
	\label{eq:linearisert}
\end{subequations}

Writing out \cref{eq:linearisert} we get

\begin{subequations}
	\begin{align}
		\ddot{\tilde p} &= \frac{L_1}{J_p} \tilde V_d = K_1 \tilde V_d\\
		\ddot{\tilde e} &= \frac{L_3}{J_e} \tilde V_s = K_2 \tilde V_s\\
		\ddot{\tilde \lambda} &= -\frac{L_2}{J_{\lambda}} \tilde p = K_3 \tilde p
	\end{align}
\end{subequations}

