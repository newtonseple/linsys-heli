\section{Part 3: Multivariable Control}
\subsection{Problem 1}
The system can be described as
\begin{subequations}
\begin{equation}
	\mbf{\dot x = Ax + Bu}
\end{equation}
where \\
	\begin{tabularx}{\textwidth}{XX}
	 \begin{equation}
	   \mbf{x} = \begin{bmatrix}
	   	\tilde p \\ \dot{\tilde p} \\ \dot{\tilde e}
	   \end{bmatrix}
	 \end{equation}
	 &
	 \begin{equation}
	   \mbf{u} = \begin{bmatrix}
	   	\tilde {V_s} \\ \tilde {V_d}
	   \end{bmatrix}
	 \end{equation}
	\\
		\begin{equation}
			\mbf{A} = \begin{bmatrix}
				0 & 1 & 0 \\ 0 & 0 & 0 \\ 0 & 0 & 0
			\end{bmatrix}
		\end{equation}
		&
		\begin{equation}
			\mbf{B} = \begin{bmatrix}
				0 & 0 \\ 0 & K_1 \\ K_2 & 0
			\end{bmatrix}
		\end{equation}
	\end{tabularx}
\end{subequations}

\subsection{Problem 2}
We verified the controllability of the system by checking that the controllability matrix $\mathcal{C}$ has full rank, as shown in \cref{eq:c}.
\begin{equation}
	\mathcal{C} = \begin{bmatrix}
		\mbf{B} & \mbf{AB} & \mbf{A^2 B}
	\end{bmatrix} = \begin{bmatrix}
		0 & 0 & 0 & K_1 & 0 & 0\\ 0 & K_1 & 0 & 0 & 0 & 0\\ K_2 & 0 & 0 & 0 & 0 & 0
	\end{bmatrix} \label{eq:c}
\end{equation}
The feed-forward matrix $\mbf{P}$ is given by %\todo{gjorde vi noen utregninger her? Kanskje sitere forelesningen/boken/?????}
\begin{equation}
	\mbf{P = (C (B K-A)^{-1} B)^{-1}}
\end{equation}
and is computed as such by MATLAB after $\mbf{K}$.

We then set out to define the weighting matrices $\mbf{Q}$ and $\mbf{R}$. We used Bryson's rule (setting each diagonal weight to the inverse square of the maximum allowable error($\mbf{Q}$) or input($\mbf{R}$)) as a starting point. Keeping the form of Bryson's rule, we adjusted the ``maximum allowable values'' until results were satisfactory. We ended up with Q and R matrices shown in \cref{eq:QR}.

\begin{subequations}
	\begin{tabularx}{\textwidth}{XX}
	 \begin{equation}
	   \mbf{Q} = \begin{bmatrix}
	   	\frac{\pi}{4}^{-2} & 0 & 0 \\
	   	0 & 2\pi^{-2} & 0 \\
	   	0&0& \frac{\pi}{5}^{-2}
	   \end{bmatrix}
	 \end{equation}
	 &
	 \begin{equation}
	   \mbf{R} = \begin{bmatrix}
	   	9^{-2} & 0 \\
	   	0 & 9^{-2}
	   \end{bmatrix}
	 \end{equation}
	\end{tabularx}
	\label{eq:QR}
\end{subequations}

Keeping the units in mind, these values make sense. The allowable pitch error is 1/8th of a circle, the allowable pitch rate ``error'' is 1 revolution per second, and the allowable elevation rate error is 1/10th of a revolution per second. The allowable inputs are 9 volts, which makes sense, considering that the motors saturate at a sum or difference of 10 volts.

Finally, the $\mbf{K}$ was calculated by the MATLAB function \texttt{K = lqr(A,B,Q,R)}. Numerically, 


\begin{subequations}
	\begin{tabularx}{\textwidth}{XX}
	 \begin{equation}
	   \mbf{K} = \begin{bmatrix}
	   	0&	0&	14.32\\
		11.46&	6.43&	0
	   \end{bmatrix}
	 \end{equation}
	 &
	 \begin{equation}
	   \mbf{P} = \begin{bmatrix}
	   	0&	14.32 \\
		11.46&	0
	   \end{bmatrix}
	 \end{equation}
	\end{tabularx}
\end{subequations}

\subsection{Problem 3}
We add the integrated states and get the augmented system with new matrices, described by the state space model
\begin{subequations}
\begin{equation}
	\mbf{\dot x_{aug} = A_{aug}x_{aug} + Bu}
\end{equation}
where \\
	\begin{tabularx}{\textwidth}{XX}
	 \begin{equation}
	   \mbf{x_{aug}} = \begin{bmatrix}
	   	\tilde p \\ \dot{\tilde p} \\ \dot{\tilde e} \\ \int_0^t{\tilde p} \\ \int_0^t{\dot{\tilde e}}
	   \end{bmatrix}
	 \end{equation}
	 &
	 \begin{equation}
	   \mbf{u} = \begin{bmatrix}
	   	\tilde {V_s} \\ \tilde {V_d}
	   \end{bmatrix}
	 \end{equation}
	\\
		\begin{equation}
			\mbf{A_{aug}} = \begin{bmatrix}
				0 & 1 & 0 & 0 & 0 \\ 0 & 0 & 0 & 0 & 0 \\ 0 & 0 & 0 & 0 & 0 \\ 1 & 0 & 0 & 0 & 0 \\ 0 & 0 & 1 & 0 & 0
			\end{bmatrix}
		\end{equation}
		&
		\begin{equation}
			\mbf{B_{aug}} = \begin{bmatrix}
				0 & 0 \\ 0 & K_1 \\ K_2 & 0 \\ 0 & 0 \\ 0 & 0 
			\end{bmatrix}
		\end{equation}
	\end{tabularx}
\end{subequations}

The new $\mbf{Q}$ and $\mbf{R}$ matrices became

\begin{subequations}
	 \begin{align}
	   \mbf{Q} &= \begin{bmatrix}
	   	\frac{\pi}{4}^{-2} & 0 & 0 & 0 & 0\\
	   	0 & 2\pi^{-2} & 0 & 0 & 0\\
	   	0&0& \frac{\pi}{5}^{-2} & 0 & 0\\
	   	0&0&0& \frac{\pi}{4}^{-2} & 0\\
	   	0&0&0&0& \frac{\pi}{5}^{-2} \\
	   \end{bmatrix}\\
	   \mbf{R} &= \begin{bmatrix}
	   	9^{-2} & 0 \\
	   	0 & 9^{-2}
	   \end{bmatrix}
	 \end{align}
\end{subequations}

$\mbf{K}$ was computed as in problem 2, but using the augmented matrices. $\mbf{P}$ was computed as in problem 2, but using only the non-augmented system and the part of K concerned with the non-integral part (including the integral part makes the calculation impossible, presumably because the system would always approach the setpoint). This was achieved through the MATLAB command \texttt{P=inv((C/(B*K(:,1:3,:)-A))*B)}. Numerically, 

\begin{subequations}
	 \begin{align}
	   \mbf{K} &= \begin{bmatrix}
	   	0		&0		&22.60	&0		&14.32 \\
		17.69	&7.921	&0		&11.46	&0
	   \end{bmatrix} \\
	   \mbf{P} &= \begin{bmatrix}
	   	0&	14.32 \\
		11.46&	0
	   \end{bmatrix}
	 \end{align}
\end{subequations}

This new controller proved even more accurate than the LQR without integral states; it is able to stay on an elevation level outside of the equilibrium. However, this introduced the danger of a runaway integral. This became apparent when the motor inputs were saturated, leading to situations where recovery was difficult, and a crash nigh unavoidable. This was mitigated by adding a limit to the setpoint, such that it was impossible to set a positive pitch rate reference at a maximum height, and impossible to set a negative pitch rate at ground level. Later (in part 4) a different approach was used, namely limiting the integrators themselves. This provided smoother control near the edges.